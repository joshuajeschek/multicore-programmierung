\documentclass{article}
\usepackage{minted}

\begin{document}

{\LARGE\textbf{Übung 5}}

\section{konkrete Schleifentransformation}
\begin{minted}[escapeinside=||]{c}
  for(i=1; i|$\leq$|2; i++)
    for(j=1; j|$\leq$|3; j++)
      for(k=1; k|$\leq$|2; k++)
        for(h=1; h|$\leq$|3; h++)
          A[i][j][k][h] = fun(i,j,k,h);
\end{minted}

\subsection{Verschmolzene Schleife}
\begin{minted}[escapeinside=||]{c}
  for (l=1;l|$\leq$|36;l++)
    i = |$\left\lceil{}\frac{l}{18}\right\rceil{} - 2 \left\lfloor{}\frac{l-1}{36}\right\rfloor{}$|
    j = |$\left\lceil{}\frac{l}{6}\right\rceil{} - 3 \left\lfloor{}\frac{l-1}{18}\right\rfloor{}$|
    k = |$\left\lceil{}\frac{l}{3}\right\rceil{} - 2 \left\lfloor{}\frac{l-1}{6}\right\rfloor{}$|
    h = |$l-3\left\lfloor{}\frac{l-1}{3}\right\rfloor{}$|
    A[i][j][k][h] = fun(i,j,k,h);
\end{minted}

\subsection{Prozessorzugriff}
\begin{itemize}
  \item $P=3$
  \item $N=36$
  \item jeder Prozessor übernimmt $r=\left\lceil{}\frac{36}{3}\right\rceil{} = 12$ Iterationen
\end{itemize}
\begin{align}
  p=1: A[1][1\dots2][1\dots2][1\dots3] \\
  p=2: A[1\dots2][3\dots1][1\dots2][1\dots3] \\
  p=3: A[2][2\dots3][1\dots2][1\dots3] \\
\end{align}

\newpage
\section{Gemischter Schleifenkomplex}
\begin{minted}{c}
  for(i=0; i<N4; i++) // parallel
    for(j=0; j<N3; j++)
      for(k=0; k<N2; k++) // parallel
        for(h=0; h<N1; h++)
          A[i][j][k][h] = A[i][j+1][k][h+1]
\end{minted}
transformiert:
\begin{minted}[escapeinside=||]{c}
  for(j=0; j<N3; j++)
    for(h=0; h<N1; h++)
      forall(l=1; l|$\leq$|N4*N2+1; l++)
        i = |$\left\lceil\frac{l}{N2}\right\rceil - N4 \left\lfloor\frac{l-1}{N2\times{}N4}\right\rfloor - 1$|
        k = |$l-N2 \left\lfloor\frac{l-1}{N2}\right\rfloor - 1$|
        A[i][j][k][h] = A[i][j+1][k][h+1]
\end{minted}


\end{document}
